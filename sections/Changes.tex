\chapter{Änderungen zum Entwurf}
    
    \section{Einleitung}
        Dokumentation über Änderungen am Entwurf
    
    \section{Änderungen am Client}
    
    \section{Änderungen am Server}
        Serverseitig wurde das System um das Modul utils.py erweitert, das grundsätzlich Funktionen für die Arbeit mit WPS Responses und der Datenbank gedacht ist. Funktionen sind leicht erweiterbar und können an anderen Stellen des Systems benutzt werden. \newline
        
        
        
        \subsection{Änderungen an REST API}
        
		\begin{center}
			\setlength\tabcolsep{5pt}
			\renewcommand{\arraystretch}{1.5}

            \begin{tabularx}{\textwidth}{|l|X|}
    			\hline
    			\rowcolor[gray]{0.90}[4.85pt]
    			Method & Änderung \\ \hline
    			GET /user & Handler -> UserView.get \\ \hline
    			GET /workflow\_refresh/:id & Neu hinzugefügt. Aktualisiert den Status des Workflows und der Tasks entsprechnend dem aktuellen Stand auf dem WPS Server \\ \hline
    			POST /process & Rausgenommen \\ \hline
    			PATCH /process/:id & Rausgenommen \\ \hline
    			DELETE /process/:id & Rausgenommen \\ \hline
    			POST /login & Neu hinzugefügt. Loggt den Benutzer ein  \\ \hline
    		\end{tabularx}
		\end{center}
	
	
%%% KLASSE CRON	
	\newpage
    	\subsection{edu.kit.scc.pseworkflow.cron}
    	Hier wird nicht django-cron sondern django-crontab verwendet, da dies wesentlich bequemer zu implementieren war und unseren Ansprüchen besser  entsprochen hat. \\
    	
    			Methoden:
		\begin{center}
			\setlength\tabcolsep{5pt}
			\renewcommand{\arraystretch}{1.5}
			
			\begin{tabularx}{\textwidth}{|l|l|l|X|}
				\hline
				\rowcolor[gray]{0.90}[4.85pt]
				Name & Rückgabetyp & Parameter & Beschreibung \\ \hline
				schedule & void & - & Hauptmethode der schedule Funktion. Hier wird der Workflow entsprechend der Reihenfolge der Tasks ausgeführt. Die XML Dateien werden erstellt und dem Servern zur Ausführung geschickt \\ \hline
				xml\_generator & void & string & Generiert die XML Datei für die auszuführede Tasks in dem übergebenen Pfad \\ \hline
				create\_data\_doc & \makecell{lxml.etree.\\\_Element/int} & task & Hilfsmethode für xml\_generator \\ \hline
				send\_task & void & \makecell{task\_id, \\xml\_dir} & Hilfsmethode für schedule. Sended Task an den WPS Server\\ \hline
				get\_execute\_url & string & task & Hilfsmethode für send\_task. Holt die Ausführungs URL aus der Datenbank. \\ \hline
				receiver & void & - & Hauptmethode der receiver Funktion. Hier werden die Ausführungsstatus der laufende Tasks überprüft, und eventuelle Ergebnisse in die Datenbank geschrieben. \\ \hline 
				parse\_execute\_response & task & int & Hilfsmethode für receiver. Prüft die Antwort XML vom WPS Server des Tasks \\ \hline
				parse\_output & void & \makecell{output, \\ task} & Hilfsmethode für parse\_execute\_response \\ \hline
				parse\_response\_literaldata & void & \makecell{artefact, \\data\_elem} & Hilfsmethode für parse\_output \\ \hline
				parse\_response\_bbox & void & \makecell{artefact, \\data\_elem} & Hilfsmethode für parse\_output \\ \hline
				parse\_response\_complexdata & void & \makecell{artefact, \\data\_elem} & Hilfsmethode für parse\_output \\ \hline
				calculate\_percent\_done & int & workflow & Berechnet den Prozentsatz fertig des Workflows \\ \hline
				task\_failed\_handling & void & \makecell{task, \\ err\_msg} & Erzeugt ein ERROR Artefact, falls der Task fehlgeschlagen ist um den Client zu informieren \\ \hline
				update\_wps\_processes & void & - & Hauptmethode um WPS Prozesse der vorhandenen WPS Server zu aktualisieren \\ \hline
			\end{tabularx}
		\end{center}
%%% ENDE DER KLASSE CRON

	
%%% MODELS
	\newpage

%%% KLASSE WORKFLOW
		\subsection{edu.kit.scc.pseworkflow.models}
			\subsubsection{Workflow}
					
			Attribute:
			\begin{center}
				\renewcommand{\arraystretch}{1.5}
				\setlength\tabcolsep{5pt}
				\begin{tabularx}{\textwidth}{|l|l|l|X|}
					\hline
					\rowcolor[gray]{0.75}[4.85pt]					
					Name & Änderung & Datentyp & Beschreibung \\ \hline
					shared & hinzugefügt & boolean & Zeigt an ob der Workflow öffentlich ist oder privat ist\\ \hline
				\end{tabularx}
			\end{center}

%%% ENDE DER KLASSE WORKFLOW

%%% ENUMERATION EXECUTION STATUS
		\subsubsection{Status (Enum)}	
			\begin{center}
				\renewcommand{\arraystretch}{1.5}
				\setlength\tabcolsep{5pt}
				\begin{tabularx}{\textwidth}{|l|l|X|}
					\hline
					\rowcolor[gray]{0.75}[4.85pt]
					Name & Änderung & Beschreibung \\ \hline
					NONE & hinzugefügt & Der Task ist idle, bzw der Workflow wird vom Nutzer bearbeitet \\ \hline
					READY & verändert & In diesem Zustand wartet ein Task auf den Input eines noch nicht beendeten Tasks  \\ \hline
					WAITING & verändert & Der Task ist bereit ausgeführt zu werden \\ \hline
				\end{tabularx}
			\end{center}
%%% ENDE DER ENUMERATION EXECUTION STATUS

%%% KLASSE WPS
		\subsubsection{WPS}
			
			Attribute:
			\begin{center}
				\setlength\tabcolsep{5pt}
				\renewcommand{\arraystretch}{1.5}
				
				\begin{tabularx}{\textwidth}{|l|l|l|l|X|}
					\hline
					\rowcolor[gray]{0.75}[4.85pt]
					Name & Änderung & Ersetzt & Rückgabetyp & Beschreibung \\ \hline 
					service\_provider & umbenannt & wps\_provider & WPSProvider & Der Besitzer des WPS Servers \\ \hline
					
				\end{tabularx}
			\end{center}
%%% ENDE DER KLASSE WPS

%%% KLASSE INPUTOUTPUT
		\subsubsection{InputOutput} 
		Die Klasse InputOutput ist nicht mehr abstakt und übernimmt die Rolle der Klassen Input und Output, welche ursprünglich von InputOutput erben sollten. Die Klassen Input und Output wurden gestrichen. \\
			Attribute:
			\begin{center}
				\setlength\tabcolsep{5pt}
				\renewcommand{\arraystretch}{1.5}
				
				\begin{tabularx}{\textwidth}{|l|l|l|X|}
					\hline
					\rowcolor[gray]{0.75}[4.85pt]
					Name & Änderung & Rückgabetyp & Beschreibung \\ \hline 
					format & hinzugefügt & string & Das Format eines inputs / outputs \\ \hline
				\end{tabularx}
			\end{center}
			
%%% ENDE DER KLASSE INPUTOUTPUT


%%% KLASSE ARTEFACT
		\subsubsection{Artefact} Die Klasse Artefact ist nicht mehr abstakt und übernimmt die Rolle der Klassen InputArtefact und OutputArtefact, welche ursprünglich von der abstrakten Klassse Artefact erben sollten. Die Klassen InputArtefact und OutputArtefact wurden gestrichen. \\

			Attribute:
			\begin{center}
				\setlength\tabcolsep{5pt}
				\renewcommand{\arraystretch}{1.5}
				
				\begin{tabularx}{\textwidth}{|l|l|l|X|}
					\hline
					\rowcolor[gray]{0.75}[4.85pt]
					Name & Änderung  & Rückgabetyp & Beschreibung \\ \hline 
					parameter & hinzugefügt & InputOutput & InputOutput Parameter des Tasks \\ \hline
				\end{tabularx}
			\end{center}
%%% ENDE DER KLASSE ARTEFACT
