\chapter{Änderungen zum Entwurf}
    
    \section{Änderungen am Client}
        Die Architektur des Clients nach der Implementierung entspricht der geplanten Architektur aus dem Entwurfsdokument. Auch die Anzahl der Klassen weicht nur leicht vom Entwurfsdokument ab. Die größte Abweichung liegt in der Komplexität der Klassen. Einige Methoden und Attribute sind neu dazugekommen, die nur schwer im Voraus zu planen waren.
        
        \subsection{Neue Komponenten}
            \subsubsection{ProcessDialogComponent}
                Eine Detailansicht des ausgewählten Prozesses. \\
                
    			Attribute:
    			\begin{center}
    				\renewcommand{\arraystretch}{1.5}
    				\setlength\tabcolsep{5pt}
    				\begin{tabularx}{\textwidth}{|l|l|X|}
    					\hline
    					\rowcolor[gray]{0.90}[4.85pt]	
    					Name & Datentyp & Beschreibung \\ \hline
    					process & Process & Der angezeigte Prozess \\ \hline
    				\end{tabularx}
    			\end{center}
    			
    	    \subsubsection{ResultDialogComponent}
                Eine Auflistung der Resultate der aktuellen Ausführung. \\
                
    			Attribute:
    			\begin{center}
    				\renewcommand{\arraystretch}{1.5}
    				\setlength\tabcolsep{5pt}
    				\begin{tabularx}{\textwidth}{|l|l|X|}
    					\hline
    					\rowcolor[gray]{0.90}[4.85pt]		
    					Name & Datentyp & Beschreibung \\ \hline
    					workflow & Workflow & Ausgeführter Workflow \\ \hline
    					results & Artefact<'output'>[] & Liste aller Resultate \\ \hline
    				\end{tabularx}
    			\end{center}
    			
    	 \subsection{Erweiterte Komponenten}
    	 
    	 % EditorComponent %
    	 \subsubsection{EditorComponent}
			Hinzugefügte Attribute:
			\begin{center}
				\setlength\tabcolsep{5pt}
				\renewcommand{\arraystretch}{1.5}
				
				\begin{tabularx}{\textwidth}{|l|l|X|}
					\hline
					\rowcolor[gray]{0.90}[4.85pt]
					Name & Datentyp & Beschreibung \\ \hline 
					movement & MovementData & Metadaten zur Mausbewegung \\ \hline
					snapshots & Workflow[] & Änderungsverlauf des Workflows \\ \hline
                    taskComponents & QueryList<TaskComponent> & Liste der Tasks im aktuellen Workflow \\ \hline
					workflowChanged & EventEmitter<Workflow>() & Wird bei jeder Änderung des Workflows ausgeführt \\ \hline
					running & boolean & Gibt an, ob sich der aktuelle Workflow in der Ausführung befindet \\ \hline
				\end{tabularx}
			\end{center}
			Hinzugefügte Methoden:
			\begin{center}
    			\setlength\tabcolsep{5pt}
    			\renewcommand{\arraystretch}{1.5}
    			
    			\begin{tabularx}{\textwidth}{|l|l|l|X|}
    				\hline
    				\rowcolor[gray]{0.90}[4.85pt]
    				Name & Rückgabetyp & Parameter & Beschreibung \\ \hline

    				clickEdge & void & edge: Edge & Wird bei klick auf eine Edge aufgerufen \\ \hline
    				scrollToMiddle & void & - & Zentriert das Sichtfeld des Editors \\ \hline
    				changeArtefact & void & - & Wird aufgerufen, wenn ein Artefact geändert wurde \\ \hline
    				snapshot & void & - & Erstellt ein Snapshot \\ \hline
    				undo & void & - & Ziegt den letzten Snapshot an \\ \hline
    			 \hline
    			\end{tabularx}
		    \end{center}
		    
    % ProcessComponent %
    	 \subsubsection{ProcessComponent}
			Hinzugefügte Methoden:
			\begin{center}
    			\setlength\tabcolsep{5pt}
    			\renewcommand{\arraystretch}{1.5}
    			\begin{tabularx}{\textwidth}{|l|l|l|X|}
    				\hline
    				\rowcolor[gray]{0.90}[4.85pt]
    				Name & Rückgabetyp & Parameter & Beschreibung \\ \hline
    				openDialog & void & - & Öffnet die Detailansicht \\ \hline
    			 %\hline
    			\end{tabularx}
		    \end{center}
		    

    % ProcessListComponent %
    	 \subsubsection{ProcessListComponent}
			Hinzugefügte Methoden:
			\begin{center}
    			\setlength\tabcolsep{5pt}
    			\renewcommand{\arraystretch}{1.5}
    			\begin{tabularx}{\textwidth}{|l|l|l|X|}
    				\hline
    				\rowcolor[gray]{0.90}[4.85pt]
    				Name & Rückgabetyp & Parameter & Beschreibung \\ \hline
    				processByWPS & void & id & Gibt alle Prozesse eines bestimmten WPS Servers zurück \\ \hline
    			 \hline
    			\end{tabularx}
		    \end{center}
		    
    % TaskComponent %
    	 \subsubsection{TaskComponent}
			Hinzugefügte Attribute:
			\begin{center}
				\setlength\tabcolsep{5pt}
				\renewcommand{\arraystretch}{1.5}
				
				\begin{tabularx}{\textwidth}{|l|l|X|}
					\hline
					\rowcolor[gray]{0.90}[4.85pt]
					Name & Datentyp & Beschreibung \\ \hline 
					parameterDrag & EventEmitter & Hilfsfunktion für Drag and Drop \\ \hline
					parameterDrop & EventEmitter & Hilfsfunktion für Drag and Drop \\ \hline
                    taskRemove & EventEmitter & Wird beim Löschen des Tasks aufgerufen \\ \hline
					changeArtefact & EventEmitter & Wird beim Ändern eines Artefacts aufgerufen \\ \hline
					running & boolean & Gibt an, ob sich der Task in der Ausführung befindet \\ \hline
				\end{tabularx}
			\end{center}
			Hinzugefügte Methoden:
			\begin{center}
    			\setlength\tabcolsep{5pt}
    			\renewcommand{\arraystretch}{1.5}
    			\begin{tabularx}{\textwidth}{|l|l|l|X|}
    				\hline
    				\rowcolor[gray]{0.90}[4.85pt]
    				Name & Rückgabetyp & Parameter & Beschreibung \\ \hline

    				hostContextmenu & void & - & Öffnet das Infomenü \\ \hline
    				openDetail & void & - & Öffnet die Detailansicht \\ \hline
    				addArtefact & void & ProcessParameter & Fügt ein Artefact dem Task hinzu \\ \hline
    				removeArtefact & void & ProcessParameter & Löscht das Artefact \\ \hline
    				hasArtefact & boolean & ProcessParameter & Gibt an, ob ein Artefact definiert ist \\ \hline
    			 \hline
    			\end{tabularx}
		    \end{center}
		    
    \newpage
    
    \section{Änderungen am Server}
        Serverseitig wurde das System um das Modul utils.py erweitert, das grundsätzlich Funktionen für die Arbeit mit WPS Responses und der Datenbank gedacht ist. Funktionen sind leicht erweiterbar und können an anderen Stellen des Systems benutzt werden. \newline
        
        
        
        \subsection{Änderungen an REST API}
        
		\begin{center}
			\setlength\tabcolsep{5pt}
			\renewcommand{\arraystretch}{1.5}

            \begin{tabularx}{\textwidth}{|l|l|X|}
    			\hline
    			\rowcolor[gray]{0.90}[4.85pt]
    			Method & Änderung & Beschreibung \\ \hline
    			GET /user & Hinzugefügt & Gibt den aktuellen Benutzer zurück \\ \hline
    			POST /process & Entfernt & Prozesse werden jetzt ausschließlich über das Hinzufügen von WPS Servern erstellt \\ \hline
    			PATCH /process/:id & Entfernt & Prozesse können aus Inkonsistenzgründen nicht mehr direkt geändert werden \\ \hline
    			DELETE /process/:id & Entfernt & Prozesse können nicht direkt entfernt werden \\ \hline
    			POST /login & Hinzugefügt & Loggt den Benutzer ein  \\ \hline
    		\end{tabularx}
		\end{center}
	
	
%%% KLASSE CRON	
	\newpage
    	\subsection{edu.kit.scc.pseworkflow.cron}
    	Hier wird nicht django-cron sondern django-crontab verwendet, da dies wesentlich bequemer zu implementieren war und unseren Ansprüchen besser  entsprochen hat. \\
    	
    			Methoden:
		\begin{center}
			\setlength\tabcolsep{5pt}
			\renewcommand{\arraystretch}{1.5}
			
			\begin{tabularx}{\textwidth}{|l|l|l|X|}
				\hline
				\rowcolor[gray]{0.90}[4.85pt]
				Name & Rückgabetyp & Parameter & Beschreibung \\ \hline
				schedule & void & - & Hauptmethode der schedule Funktion. Hier wird der Workflow entsprechend der Reihenfolge der Tasks ausgeführt. Die XML Dateien werden erstellt und dem Servern zur Ausführung geschickt \\ \hline
				xml\_generator & void & string & Generiert die XML Datei für die auszuführede Tasks in dem übergebenen Pfad \\ \hline
				create\_data\_doc & \makecell{lxml.etree.\\\_Element/int} & task & Hilfsmethode für xml\_generator \\ \hline
				send\_task & void & \makecell{task\_id, \\xml\_dir} & Hilfsmethode für schedule. Sended Task an den WPS Server\\ \hline
				get\_execute\_url & string & task & Hilfsmethode für send\_task. Holt die Ausführungs URL aus der Datenbank. \\ \hline
				receiver & void & - & Hauptmethode der receiver Funktion. Hier werden die Ausführungsstatus der laufende Tasks überprüft, und eventuelle Ergebnisse in die Datenbank geschrieben. \\ \hline 
				parse\_execute\_response & task & int & Hilfsmethode für receiver. Prüft die Antwort XML vom WPS Server des Tasks \\ \hline
				parse\_output & void & \makecell{output, \\ task} & Hilfsmethode für parse\_execute\_response \\ \hline
				parse\_response\_literaldata & void & \makecell{artefact, \\data\_elem} & Hilfsmethode für parse\_output \\ \hline
				parse\_response\_bbox & void & \makecell{artefact, \\data\_elem} & Hilfsmethode für parse\_output \\ \hline
				parse\_response\_complexdata & void & \makecell{artefact, \\data\_elem} & Hilfsmethode für parse\_output \\ \hline
				calculate\_percent\_done & int & workflow & Berechnet den Prozentsatz fertig des Workflows \\ \hline
				task\_failed\_handling & void & \makecell{task, \\ err\_msg} & Erzeugt ein ERROR Artefact, falls der Task fehlgeschlagen ist um den Client zu informieren \\ \hline
				update\_wps\_processes & void & - & Hauptmethode um WPS Prozesse der vorhandenen WPS Server zu aktualisieren \\ \hline
			\end{tabularx}
		\end{center}
%%% ENDE DER KLASSE CRON

	
%%% MODELS
	\newpage

%%% KLASSE WORKFLOW
		\subsection{edu.kit.scc.pseworkflow.models}
			\subsubsection{Workflow}
					
			Attribute:
			\begin{center}
				\renewcommand{\arraystretch}{1.5}
				\setlength\tabcolsep{5pt}
				\begin{tabularx}{\textwidth}{|l|l|l|X|}
					\hline
					\rowcolor[gray]{0.9}[4.85pt]					
					Name & Änderung & Datentyp & Beschreibung \\ \hline
					shared & Hinzugefügt & boolean & Zeigt an ob der Workflow öffentlich ist oder privat ist\\ \hline
				\end{tabularx}
			\end{center}

%%% ENDE DER KLASSE WORKFLOW

%%% ENUMERATION EXECUTION STATUS
		\subsubsection{Status (Enum)}	
			\begin{center}
				\renewcommand{\arraystretch}{1.5}
				\setlength\tabcolsep{5pt}
				\begin{tabularx}{\textwidth}{|l|l|X|}
					\hline
					\rowcolor[gray]{0.9}[4.85pt]
					Name & Änderung & Beschreibung \\ \hline
					NONE & Hinzugefügt & Der Task ist idle, bzw der Workflow wird vom Nutzer bearbeitet \\ \hline
					READY & Geändert & In diesem Zustand wartet ein Task auf den Input eines noch nicht beendeten Tasks  \\ \hline
					WAITING & Geändert & Der Task ist bereit ausgeführt zu werden \\ \hline
				\end{tabularx}
			\end{center}
%%% ENDE DER ENUMERATION EXECUTION STATUS

%%% KLASSE WPS
		\subsubsection{WPS}
			
			Attribute:
			\begin{center}
				\setlength\tabcolsep{5pt}
				\renewcommand{\arraystretch}{1.5}
				
				\begin{tabularx}{\textwidth}{|l|l|l|l|X|}
					\hline
					\rowcolor[gray]{0.9}[4.85pt]
					Name & Änderung & Ersetzt & Rückgabetyp & Beschreibung \\ \hline 
					service\_provider & umbenannt & wps\_provider & WPSProvider & Der Besitzer des WPS Servers \\ \hline
					
				\end{tabularx}
			\end{center}
%%% ENDE DER KLASSE WPS

%%% KLASSE INPUTOUTPUT
		\subsubsection{InputOutput} 
		Die Klasse InputOutput ist nicht mehr abstakt und übernimmt die Rolle der Klassen Input und Output, welche ursprünglich von InputOutput erben sollten. Die Klassen Input und Output wurden gestrichen. \\
			Attribute:
			\begin{center}
				\setlength\tabcolsep{5pt}
				\renewcommand{\arraystretch}{1.5}
				
				\begin{tabularx}{\textwidth}{|l|l|l|X|}
					\hline
					\rowcolor[gray]{0.9}[4.85pt]
					Name & Änderung & Rückgabetyp & Beschreibung \\ \hline 
					format & Hinzugefügt & string & Das Format eines Inputs oder Outputs \\ \hline
				\end{tabularx}
			\end{center}
			
%%% ENDE DER KLASSE INPUTOUTPUT


%%% KLASSE ARTEFACT
		\subsubsection{Artefact} Die Klasse Artefact ist nicht mehr abstakt und übernimmt die Rolle der Klassen InputArtefact und OutputArtefact, welche ursprünglich von der abstrakten Klassse Artefact erben sollten. Die Klassen InputArtefact und OutputArtefact wurden gestrichen. \\

			Attribute:
			\begin{center}
				\setlength\tabcolsep{5pt}
				\renewcommand{\arraystretch}{1.5}
				
				\begin{tabularx}{\textwidth}{|l|l|l|X|}
					\hline
					\rowcolor[gray]{0.9}[4.85pt]
					Name & Änderung  & Rückgabetyp & Beschreibung \\ \hline 
					parameter & Hinzugefügt & InputOutput & InputOutput Parameter des Tasks \\ \hline
				\end{tabularx}
			\end{center}
%%% ENDE DER KLASSE ARTEFACT
