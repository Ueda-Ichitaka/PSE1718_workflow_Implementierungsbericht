\chapter{Änderungen zum Entwurf}
    
    \section{Einleitung}
        Dokumentation über Änderungen am Entwurf
    
    \section{Änderungen am Client}
    
    \section{Änderungen am Server}
        Serverseitig wurde das System um das Modul utils.py erweitert, das grundsätzlich Funktionen für die Arbeit mit WPS Responses und der Datenbank gedacht ist. Funktionen sind leicht erweiterbar und können an anderen Stellen des Systems benutzt werden. \newline
        InputOutput in Database
        
        complete cron
        
        
        
        \subsection{Änderungen an REST API}
        
		\begin{center}
            \begin{tabularx}{\textwidth}{|l|X|}
    			\hline
    			\rowcolor[gray]{0.90}[4.85pt]
    			Method & Änderung \\ \hline
    			GET /user & Handler -> UserView.get \\ \hline
    			GET /workflow\_refresh/:id & Neu hinzugefügt. Aktualisiert den Status des Workflows und der Tasks entsprechnend dem aktuellen Stand auf dem WPS Server \\ \hline
    			POST /process & Rausgenommen \\ \hline
    			PATCH /process/:id & Rausgenommen \\ \hline
    			DELETE /process/:id & Rausgenommen \\ \hline
    			POST /login & Neu hinzugefügt. Loggt den Benutzer ein  \\ \hline
    		\end{tabularx}
		\end{center}
	
	
	
    	\subsection{edu.kit.scc.pseworkflow.cron}
    	Hier wird nicht django-cron sondern django-crontab verwendet, da dies wesentlich bequemer zu implementieren war und unseren Ansprüchen besser  entsprochen hat.
    	
    			Methoden:
		\begin{center}
			\setlength\tabcolsep{5pt}
			\renewcommand{\arraystretch}{1.5}
			
			\begin{tabularx}{\textwidth}{|l|l|l|X|}
				\hline
				\rowcolor[gray]{0.90}[4.85pt]
				Name & Rückgabetyp & Parameter & Beschreibung \\ \hline
				schedule & void & - & Hauptmethode der schedule Funktion. Hier wird der Workflow entsprechend der Reihenfolge der Tasks ausgeführt. Die XML Dateien werden erstellt und dem Servern zur Ausführung geschickt \\ \hline
				xml\_generator & void & string & Generiert die XML Datei für die auszuführede Tasks in dem übergebenen Pfad \\ \hline
				create\_data\_doc & \makecell{lxml.etree.\\\_Element/int} & task & Hilfsmethode für xml\_generator \\ \hline
				send\_task & void & \makecell{task\_id, \\xml\_dir} & Hilfsmethode für schedule. Sended Task an den WPS Server \\ \hline
				get\_execute\_url & string & task & Hilfsmethode für send\_task. Holt die Ausführungs URL aus der Datenbank. \\ \hline
				receiver & void & - & Hauptmethode der receiver Funktion. Hier werden die Ausführungsstatus der laufende Tasks überprüft. 
			\end{tabularx}
		\end{center}