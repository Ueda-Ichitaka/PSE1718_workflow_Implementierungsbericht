\chapter{\gls{Unittest}s}
    \begin{flushleft}
    In diesem Abschnitt werden alle Unittests des Systems aufgelistet und kurz beschrieben. \newline
    Es ist von Django framework verlangt, dass alle Testmethoden mit dem Präfix \glqq test\grqq \ anfangen. \newline 
    Da alle im Weiteren beschriebene Methoden Testmethoden sind, haben die alle den Rückgabetyp void und erwarten keine Eingabeparameter.  
    \end{flushleft}
    
    \section{ParserTestCase}
    
    \begin{flushleft}
    Innherhalb dieser Testklasse werden Methoden getestet, die für die Analyse der WPS Server Responses verantwortlich sind. Dabei werden zwei statische xml Testdateien verwendet, die mittels WPS getCapabilities und WPS describeProcesses Requests als Response von einem test Server zurückgegenen werden. \newline
    In der Methoden Beschreibungen wird der Einfachheit halber oft das Wort \glqq Information \grqq benutzt. Genau Liste der Attributen die für das Erzeugen eines Objekts notwendig sind, finden Sie im Entwurfsdokument (Abschnitt 3.1.3 Models).
    \end{flushleft}
	
	\begin{center}
		\renewcommand{\arraystretch}{1.5}
		\setlength\tabcolsep{5pt}
		\begin{tabularx}{\textwidth}{|l|X|}
			\hline
			\rowcolor[gray]{0.75}[4.85pt]
			Name & Beschreibung \\ \hline
			test\_parse\_service\_provider\_info & Testet, ob die Information über einen WPS Provider richtig aus einer xml Datei ausgelesen wird. \\ \hline
			test\_parse\_service\_provider\_info\_fail &  Prüft die Ausnahmebehandlung beim Parsen der Information über einen service provider. \\ \hline
			test\_parse\_wps\_server\_info & Testet, ob die Information über einen WPS Server richtig aus einer xml Datei ausgelesen wird. \\ \hline
			test\_parse\_wps\_server\_info\_fail & Prüft die Ausnahmebehandlung beim Parsen der Information über einen Server. \\ \hline
			test\_parse\_wps\_process & Es wird getestet, ob die Information über einen WPS Prozess richtig aus einer xml Datei ausgelesen wird. \\ \hline
			test\_parse\_wps\_process\_fail & Prüft die Ausnahmebehandlung beim Parsen der Information über einen WPS Prozess. \\ \hline
			test\_parse\_process\_input\_literal & Testet, ob die Information über einen Input mit Datentyp \glqq LiteralData\grqq \ korrekt ausgelesen wird.\\ \hline
			test\_parse\_process\_output\_literal & Es wird überprüft, ob die Information über einen Output mit Datentyp \glqq LiteralOutput\grqq \ korrekt ausgelesen wird. \\ \hline
	\end{tabularx}
	\end{center}	
	
	
	\begin{center}
		\renewcommand{\arraystretch}{1.5}
		\setlength\tabcolsep{5pt}
		\begin{tabularx}{\textwidth}{|l|X|}
			\hline
			\rowcolor[gray]{0.75}[4.85pt]
			Name & Beschreibung \\ \hline
			test\_parse\_process\_input\_complex & Es wird überprüft, ob die Information über einen Input mit Datentyp \glqq ComplexData\grqq \ korrekt ausgelesen wird. \\ \hline
			test\_parse\_process\_output\_complex & Es wird überprüft, ob die Information über einen Output mit Datentyp \glqq ComplexOutput\grqq \ korrekt ausgelesen wird. \\ \hline
			test\_parse\_process\_input\_bounding\_box & Es wird überprüft, ob die Information über einen Input mit Datentyp \glqq BoundingBoxData\grqq \ korrekt ausgelesen wird. \\ \hline
			test\_parse\_process\_output\_bounding\_box & Es wird überprüft, ob die Information über einen Output mit Datentyp \glqq BoundingBoxOutput\grqq \ korrekt ausgelesen wird. \\ \hline
			test\_parse\_process\_input\_fail & Prüft die Ausnahmebehandlung beim Parsen der Information über einen  Input. \\ \hline
			test\_parse\_process\_output\_fail & Prüft die Ausnahmebehandlung beim Parsen der Information über einen Output.\\ \hline
		\end{tabularx}
	\end{center}	
	
%%%%%%%%%%%
% NEWPAGE %
%%%%%%%%%%%

    \newpage

	\section{DatabaseTestCase}
	In dieser Testklasse werden Methoden getestet, die für die Arbeit mit einer relationalen Datenbank verantwortlich sind. Methoden mit dem Präfix \glqq search\grqq \ werden verwendet, um Duplikate in der Datenbank zu eliminieren.
	
	\begin{center}
		\renewcommand{\arraystretch}{1.5}
		\setlength\tabcolsep{5pt}
		\begin{tabularx}{\textwidth}{|l|X|}
			\hline
			\rowcolor[gray]{0.75}[4.85pt]
			Name & Beschreibung \\ \hline
			setUp & (Überschrieben) \newline Es werden die Testdaten in die Datenbank geschrieben. \\ \hline
			test\_searc\_provider\_in\_database & Sucht einen Service Provider in der Datenbank nach den Namen und der Webseite. \\ \hline
			test\_search\_provider\_in\_empty\_database & Prüft, ob im Falle der Abwesenheit des Providers in der Datenbank, eine Ausnahme geworfen wird. \\ \hline
			test\_search\_server\_in\_database & Sucht einen Server in der Datenbank nach den Titel. \\ \hline
			test\_search\_server\_in\_empty\_database & Prüft, ob im Falle der Abwesenheit des Servers in der Datenbank, eine Ausnahme geworfen wird. \\ \hline
			test\_search\_process\_in\_database & Sucht einen WPS Prozess in der Datenbank nach den Titel und Titel des dazugehörigen Servers.  \\ \hline
			test\_search\_process\_in\_empty\_database & Prüft, ob im Falle der Abwesenheit des WPS Prozesses in der Datenbank, eine Ausnahme geworfen wird. \\ \hline
			test\_search\_input\_output\_in\_database & Sucht einen Input/Output in der Datenbank nach seinen Identifikator (String) und den dazugehörigen WPS Prozess. \\ \hline
			test\_search\_input\_output\_in\_empty\_database & Prüft, ob im Falle der Abwesenheit des Inputs/Outputs in der Datenbank, eine Ausnahme geworfen wird. \\ \hline
			test\_overwrite\_server & Testet, ob die Information über Server korrekt überschrieben wird.\\ \hline
			test\_overwrite\_process & Testet, ob die Information über WPS Prozess korrekt überschrieben wird.\\ \hline
			test\_overwrite\_input\_output & Testet, ob die Information über Input/Output korrekt überschrieben wird.\\ \hline
		\end{tabularx}
	\end{center}
	
	
	
	\section{CronTestCase}
	Innerhalb dieser Testklasse werden Methoden des Moduls cron.py getestet, durch einen Cron daemon in regelmäßigen Abständen ausgeführt werden.
	
	\begin{center}
		\renewcommand{\arraystretch}{1.5}
		\setlength\tabcolsep{5pt}
		\begin{tabularx}{\textwidth}{|l|X|}
			\hline
			\rowcolor[gray]{0.75}[4.85pt]
			Name & Beschreibung \\ \hline
			setUp & (Überschrieben) \newline Es werden die Testdaten in die Datenbank geschrieben. \\ \hline
			tearDown & (Überschrieben) \newline Es werden alle Datensätze aus der Datenbank gelöscht\\ \hline
			test\_send\_task & Es wird ein Test Task ausgewählt und überprüft, dass man einen Response von Server bekommt und die status URL in die Datenbank geschrieben wird. \\ \hline
			test\_execution & Führt einen Task aus und überprüft, ob in der Datenbank ein richtiges Ergebnis steht. \\ \hline
			test\_update\_wps\_processes & Testet, ob die WPS Prozesse von einem im voraus gespeicherten WPS Server korrekt gesammelt und in der Datenbank gespeichert werden.\\ \hline
			test\_update\_wps\_processes\_with\_empty\_database & Testet, ob bei einer leeren Datenbank (also, wenn es kein Server gespeichert ist) nichts gemacht wird. \\ \hline
			
		\end{tabularx}
	\end{center}
	
	
%%% Mache nach dem Essen weiter